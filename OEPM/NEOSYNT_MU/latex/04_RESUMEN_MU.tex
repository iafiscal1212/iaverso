% 04_RESUMEN_MU.tex - Resumen para Modelo de Utilidad OEPM
\documentclass[11pt,a4paper]{article}

\usepackage[a-1b]{pdfx}
\usepackage[utf8]{inputenc}
\usepackage[T1]{fontenc}
\usepackage[spanish]{babel}
\usepackage{lmodern}
\usepackage{microtype}
\usepackage[left=2.5cm,right=2.5cm,top=2.5cm,bottom=2.5cm]{geometry}
\usepackage{setspace}
\setstretch{1.3}

\begin{document}

\begin{center}
{\LARGE\bfseries RESUMEN}\\[1cm]
{\large Dispositivo de control autónomo multi-agente\\
con bus local y núcleo endógeno de seguridad (NEOSYNT)}\\[2cm]
\end{center}

\noindent\textbf{RESUMEN}

\vspace{0.5cm}

Se divulga un dispositivo de control autónomo multi-agente que comprende: (i) un bus local por socket UNIX que intercambia resúmenes estadísticos entre agentes; (ii) buffers circulares por agente; (iii) un módulo de validación con checksum SHA-256 y registro inmutable con hash encadenado; (iv) un núcleo autónomo que mantiene un vector de intención $[S, N, C]$ restringido al simplex, actualizado mediante mirror-descent con inyección de ruido Ornstein-Uhlenbeck de parámetros endógenos; (v) una puerta de consentimiento bilateral que integra urgencia, primer autovalor, confianza y coeficiente de variación; (vi) un watchdog de recursos con umbrales dinámicos; y (vii) un sandbox de evolución de código condicionado por umbrales endógenos de estabilidad. La combinación estructural reduce colapsos en un 87\% y latencia en un 45\%, aumentando la robustez frente a ruido respecto a configuraciones sin dichos elementos. Aplicaciones: orquestación edge, robótica cooperativa, gestión de colas y caché.

\vspace{1cm}

\noindent\textbf{Figura representativa:} Figura 1

\vspace{1cm}

\noindent\textbf{Clasificación IPC:} G06N 3/00; G06N 20/00; G06F 9/50; G06F 15/18

\vspace{1cm}

\noindent\textbf{Palabras clave:} control multi-agente, bus local, simplex, mirror-descent, Ornstein-Uhlenbeck endógeno, consentimiento bilateral, watchdog, sandbox, robustez

\end{document}
