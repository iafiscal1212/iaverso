% 02_REIVINDICACIONES_MU.tex - Reivindicaciones para Modelo de Utilidad OEPM
\documentclass[11pt,a4paper]{article}

\usepackage[a-1b]{pdfx}
\usepackage[utf8]{inputenc}
\usepackage[T1]{fontenc}
\usepackage[spanish]{babel}
\usepackage{lmodern}
\usepackage{microtype}
\usepackage[left=2.5cm,right=2.5cm,top=2.5cm,bottom=2.5cm]{geometry}
\usepackage{amsmath}
\usepackage{amssymb}
\usepackage{setspace}
\setstretch{1.2}
\usepackage{enumitem}

\usepackage{fancyhdr}
\pagestyle{fancy}
\fancyhf{}
\rhead{NEOSYNT - Modelo de Utilidad}
\lhead{Reivindicaciones}
\rfoot{Página \thepage}

\begin{document}

\begin{center}
{\LARGE\bfseries REIVINDICACIONES}\\[1cm]
{\large Dispositivo de control autónomo multi-agente\\
con bus local y núcleo endógeno de seguridad (NEOSYNT)}\\[2cm]
\end{center}

\section*{Reivindicaciones}

\begin{enumerate}[label=\textbf{\arabic*.}, leftmargin=*, labelwidth=2em]

% ============================================================================
% REIVINDICACIÓN 1 - INDEPENDIENTE (PRINCIPAL)
% ============================================================================
\item \textbf{Dispositivo de control autónomo multi-agente}, que comprende:

\begin{enumerate}[label=(\alph*)]
    \item un bus local (100) configurado como socket UNIX para intercambiar resúmenes estadísticos entre agentes;

    \item buffers circulares (110) por agente para almacenamiento temporal de estados y mensajes;

    \item un módulo de validación (120) con checksum y registro inmutable de mensajes;

    \item un núcleo autónomo (130) configurado para mantener y actualizar, por agente, un vector de intención $\mathbf{I} = [S, N, C]$ restringido al simplex, y modular la interacción mediante una puerta de consentimiento bilateral (140);

    \item un watchdog de recursos (150) con umbrales derivados endógenamente de estadísticas históricas;

    \item un sandbox de evolución de código (160) condicionado a umbrales endógenos de estabilidad;
\end{enumerate}

en donde la configuración (a)-(f) reduce colapsos del sistema y aumenta robustez frente a ruido respecto a dispositivos sin dicha combinación.

% ============================================================================
% REIVINDICACIONES DEPENDIENTES
% ============================================================================

\item El dispositivo de la reivindicación 1, en donde el bus local (100) es un socket UNIX de tipo datagrama (SOCK\_DGRAM) con tamaño máximo de mensaje preestablecido.

\item El dispositivo de la reivindicación 1 o 2, en donde los buffers circulares (110) implementan descarte FIFO con longitud máxima (maxlen) configurable derivada de $\lceil \sqrt{t+1} \times k \rceil$, donde $t$ es el tiempo de operación y $k$ un factor estructural.

\item El dispositivo de cualquiera de las reivindicaciones anteriores, en donde el módulo de validación (120) aplica:
\begin{itemize}
    \item checksum SHA-256 parcial sobre los primeros bytes de cada mensaje;
    \item sello temporal de alta resolución; y
    \item hash encadenado en el registro inmutable, donde cada entrada incluye el hash de la entrada anterior.
\end{itemize}

\item El dispositivo de cualquiera de las reivindicaciones anteriores, en donde el núcleo autónomo (130) actualiza el vector de intención $\mathbf{I}$ mediante mirror-descent en espacio logit con proyección softmax al simplex.

\item El dispositivo de cualquiera de las reivindicaciones anteriores, en donde el núcleo autónomo (130) inyecta ruido Ornstein-Uhlenbeck con parámetros $(\theta, \sigma, \tau)$ estimados endógenamente a partir de:
\begin{itemize}
    \item $\theta$: autocorrelación de los residuos del sistema;
    \item $\sigma$: varianza de los residuos;
    \item $\tau = 1/\theta$: tiempo de correlación.
\end{itemize}

\item El dispositivo de cualquiera de las reivindicaciones anteriores, en donde la puerta de consentimiento bilateral (140) integra las siguientes variables para determinar el consentimiento de cada agente:
\begin{itemize}
    \item urgencia ($u$), derivada de la componente de conexión del vector de intención;
    \item primer autovalor ($\lambda_1$) del análisis de componentes principales del historial;
    \item confianza (conf), basada en historial de interacciones exitosas;
    \item coeficiente de variación (CV) del error de predicción;
\end{itemize}
permitiendo una interacción solo cuando ambos agentes consienten simultáneamente.

\item El dispositivo de cualquiera de las reivindicaciones anteriores, en donde el watchdog de recursos (150) aplica umbrales dinámicos derivados de:
\begin{equation*}
\text{threshold}_i(t) = \text{percentile}_{95}(\text{history}_i[t-w:t])
\end{equation*}
donde $w$ es una ventana temporal endógena para cada recurso monitorizado (CPU, RAM, I/O).

\item El dispositivo de cualquiera de las reivindicaciones anteriores, en donde el sandbox de evolución de código (160) solo se activa cuando:
\begin{equation*}
S > 0.6 \quad \land \quad \text{stability\_index} > 0.6
\end{equation*}
siendo $S$ la componente de estabilidad del vector de intención y stability\_index un índice derivado de la varianza del estado interno.

\item El dispositivo de cualquiera de las reivindicaciones anteriores, que comprende además un planificador (170) con horizonte adaptativo:
\begin{equation*}
h = \lceil \log_2(t+1) \rceil
\end{equation*}
para gestión de colas y caché, donde $t$ es el tiempo de operación.

\item El dispositivo de cualquiera de las reivindicaciones anteriores, en donde los mensajes intercambiados por el bus local (100) contienen únicamente resúmenes estadísticos (media, varianza, percentiles, hash de estado) y no vectores completos de estado interno, limitando así la exposición de información y el consumo de ancho de banda.

\item Dispositivo según cualquiera de las reivindicaciones anteriores, caracterizado por proporcionar:
\begin{itemize}
    \item una reducción $\geq 85\%$ de colapsos del sistema; y
    \item una disminución $\geq 40\%$ de latencia de coordinación;
\end{itemize}
frente a una configuración nula (sin los elementos (100)-(170)) en pruebas estandarizadas con al menos 1000 ciclos de operación y condiciones de ruido controladas.

\end{enumerate}

\vfill

\begin{center}
\rule{0.5\textwidth}{0.4pt}\\[0.5cm]
{\small Fin de las reivindicaciones}
\end{center}

\end{document}
