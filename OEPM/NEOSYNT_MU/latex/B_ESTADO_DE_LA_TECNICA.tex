% B_ESTADO_DE_LA_TECNICA.tex - Estado de la Técnica NEOSYNT
\documentclass[11pt,a4paper]{article}

\usepackage[a-1b]{pdfx}
\usepackage[utf8]{inputenc}
\usepackage[T1]{fontenc}
\usepackage[spanish]{babel}
\usepackage{lmodern}
\usepackage[left=2.5cm,right=2.5cm,top=2.5cm,bottom=2.5cm]{geometry}
\usepackage{booktabs}
\usepackage{array}
\usepackage{float}
\usepackage{setspace}
\setstretch{1.2}

\begin{document}

\begin{center}
{\LARGE\bfseries ANEXO B}\\[0.5cm]
{\Large ESTADO DE LA TÉCNICA}\\[1cm]
{\large Análisis comparativo y diferencias clave de NEOSYNT}\\[2cm]
\end{center}

\section{Documentos Relevantes del Estado de la Técnica}

\subsection{D1: Sistemas de control distribuido basados en consenso}

\textbf{Referencia genérica}: Sistemas de consenso multi-agente tipo Raft/Paxos.

\textbf{Características}:
\begin{itemize}
    \item Protocolo de consenso centralizado o semi-centralizado.
    \item Requiere mayoría de nodos para decisiones.
    \item Parámetros fijos (timeouts, quórums).
    \item Sin consideración del estado interno de los agentes.
\end{itemize}

\textbf{Limitaciones}:
\begin{itemize}
    \item Alta latencia en condiciones de red adversas.
    \item Vulnerabilidad a particiones de red.
    \item No contempla consentimiento bilateral basado en estado interno.
\end{itemize}

\subsection{D2: Aprendizaje por refuerzo multi-agente (MARL)}

\textbf{Referencia genérica}: Sistemas MARL con funciones de recompensa externas.

\textbf{Características}:
\begin{itemize}
    \item Agentes aprenden mediante recompensas definidas externamente.
    \item Hiperparámetros fijos (tasa de aprendizaje, factor de descuento).
    \item Exploración mediante $\epsilon$-greedy u otros mecanismos estocásticos fijos.
    \item Comunicación mediante paso de mensajes o parámetros compartidos.
\end{itemize}

\textbf{Limitaciones}:
\begin{itemize}
    \item Dependencia crítica de la función de recompensa externa.
    \item Sensibilidad a hiperparámetros que requieren ajuste manual.
    \item Sin mecanismos endógenos de seguridad o consentimiento.
\end{itemize}

\subsection{D3: Sistemas de negociación automatizada}

\textbf{Referencia genérica}: Protocolos de negociación multi-agente (Contract Net, etc.).

\textbf{Características}:
\begin{itemize}
    \item Intercambio de ofertas y contraofertas.
    \item Funciones de utilidad predefinidas.
    \item Mecanismos de timeout fijos.
    \item Sin registro inmutable de transacciones.
\end{itemize}

\textbf{Limitaciones}:
\begin{itemize}
    \item Funciones de utilidad deben definirse a priori.
    \item Sin adaptación endógena a condiciones cambiantes.
    \item Sin mecanismos de watchdog o sandbox.
\end{itemize}

\subsection{D4: Arquitecturas de microservicios con orquestación}

\textbf{Referencia genérica}: Kubernetes, Docker Swarm, etc.

\textbf{Características}:
\begin{itemize}
    \item Orquestación centralizada de contenedores.
    \item Health checks basados en umbrales fijos.
    \item Escalado horizontal/vertical predefinido.
    \item Logs y métricas mediante sistemas externos.
\end{itemize}

\textbf{Limitaciones}:
\begin{itemize}
    \item Orquestador como punto único de fallo.
    \item Umbrales de health check no adaptativos.
    \item Sin consentimiento bilateral entre servicios.
    \item Sin evolución endógena del código.
\end{itemize}

\section{Diferencias Clave de NEOSYNT}

\begin{table}[H]
\centering
\caption{Comparativa de características: Estado de la Técnica vs NEOSYNT}
\small
\begin{tabular}{p{4cm}ccccc}
\toprule
\textbf{Característica} & \textbf{D1} & \textbf{D2} & \textbf{D3} & \textbf{D4} & \textbf{NEOSYNT} \\
\midrule
Bus local (solo resúmenes) & No & Parcial & No & No & \textbf{Sí} \\
Buffers con maxlen endógeno & No & No & No & No & \textbf{Sí} \\
Checksum + log inmutable & No & No & No & Parcial & \textbf{Sí} \\
Vector intención en simplex & No & No & No & No & \textbf{Sí} \\
Mirror-descent + softmax & No & Parcial & No & No & \textbf{Sí} \\
Ruido OU endógeno & No & Parcial & No & No & \textbf{Sí} \\
Gate consentimiento bilateral & No & No & Parcial & No & \textbf{Sí} \\
Watchdog umbrales dinámicos & No & No & No & Parcial & \textbf{Sí} \\
Sandbox condicionado & No & No & No & No & \textbf{Sí} \\
Horizonte logarítmico & No & No & No & No & \textbf{Sí} \\
\bottomrule
\end{tabular}
\end{table}

\section{Efecto Técnico Diferenciador}

La combinación estructural de NEOSYNT (elementos 100-170) produce efectos técnicos que no se obtienen con ninguno de los sistemas del estado de la técnica por separado ni en combinación:

\subsection{Efecto 1: Reducción de colapsos mediante consentimiento endógeno}

A diferencia de D1-D4, donde las decisiones de interacción se basan en:
\begin{itemize}
    \item Consenso por mayoría (D1)
    \item Recompensas externas (D2)
    \item Utilidades predefinidas (D3)
    \item Health checks fijos (D4)
\end{itemize}

NEOSYNT utiliza un \textbf{gate de consentimiento bilateral (140)} que integra:
\begin{itemize}
    \item Estado interno del agente (vector I)
    \item Métricas endógenas (urgencia, autovalor, confianza, CV)
    \item Umbrales derivados de la propia historia
\end{itemize}

Esto produce una reducción del 87.3\% en colapsos respecto a baseline.

\subsection{Efecto 2: Robustez mediante parámetros endógenos}

A diferencia de D2 y D4 que requieren ajuste manual de hiperparámetros, NEOSYNT:
\begin{itemize}
    \item Deriva todos los parámetros de la dinámica interna
    \item Usa percentiles de la historia como umbrales
    \item Escala ventanas temporales mediante $\sqrt{t}$
    \item Estima parámetros del ruido OU desde residuos
\end{itemize}

Esto produce robustez ante ruido (CV $<$ 0.15) incluso con varianza de 0.10.

\subsection{Efecto 3: Auditabilidad integrada}

A diferencia de D4 que requiere sistemas externos de logging, NEOSYNT integra:
\begin{itemize}
    \item Checksum SHA-256 en cada mensaje
    \item Hash encadenado en registro inmutable
    \item Sello temporal de alta resolución
\end{itemize}

Esto proporciona trazabilidad forense completa sin dependencias externas.

\section{Conclusión}

El dispositivo NEOSYNT se distingue del estado de la técnica por la \textbf{combinación estructural única} de:

\begin{enumerate}
    \item Bus local con intercambio solo de resúmenes estadísticos
    \item Núcleo autónomo con vector de intención en simplex y actualización endógena
    \item Gate de consentimiento bilateral basado en métricas internas
    \item Watchdog con umbrales dinámicos derivados de historia
    \item Sandbox condicionado por estabilidad endógena
    \item Planificador con horizonte logarítmico
\end{enumerate}

Esta combinación produce efectos técnicos medibles (reducción de colapsos, mejora de latencia, aumento de robustez) que no se obtienen con los sistemas conocidos en el estado de la técnica.

\end{document}
