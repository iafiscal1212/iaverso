% A_ANEXO_DATOS_EXPERIMENTALES.tex - Datos Experimentales NEOSYNT
\documentclass[11pt,a4paper]{article}

\usepackage[a-1b]{pdfx}
\usepackage[utf8]{inputenc}
\usepackage[T1]{fontenc}
\usepackage[spanish]{babel}
\usepackage{lmodern}
\usepackage[left=2.5cm,right=2.5cm,top=2.5cm,bottom=2.5cm]{geometry}
\usepackage{graphicx}
\usepackage{booktabs}
\usepackage{array}
\usepackage{float}
\usepackage{amsmath}
\usepackage{setspace}
\setstretch{1.2}

\begin{document}

\begin{center}
{\LARGE\bfseries ANEXO A}\\[0.5cm]
{\Large DATOS EXPERIMENTALES}\\[1cm]
{\large Dispositivo de control autónomo multi-agente NEOSYNT}\\[2cm]
\end{center}

\section{Protocolo de Pruebas}

\subsection{Configuración del Sistema}

\begin{table}[H]
\centering
\caption{Configuración hardware y software de pruebas}
\begin{tabular}{ll}
\toprule
\textbf{Parámetro} & \textbf{Valor} \\
\midrule
Procesador & Intel Core i7 / AMD equivalente \\
Memoria RAM & 8 GB \\
Sistema Operativo & Linux (kernel 5.15+) \\
Lenguaje de implementación & Python 3.10+ \\
Número de agentes & 2-3 (NEO, EVA, ALEX) \\
Ciclos por experimento & 1500 \\
Repeticiones (semillas) & 10 \\
\bottomrule
\end{tabular}
\end{table}

\subsection{Condiciones de Prueba}

\begin{table}[H]
\centering
\caption{Condiciones experimentales}
\begin{tabular}{lcc}
\toprule
\textbf{Condición} & \textbf{Ruido (varianza)} & \textbf{Carga (factor)} \\
\midrule
Normal & 0.01 & 1.0x \\
Media & 0.05 & 2.0x \\
Alta & 0.10 & 5.0x \\
\bottomrule
\end{tabular}
\end{table}

\section{Resultados Experimentales}

\subsection{Comparativa Principal: NEOSYNT vs Baseline}

\begin{table}[H]
\centering
\caption{Resultados principales (n=10 semillas, 1500 ciclos)}
\begin{tabular}{lcccr}
\toprule
\textbf{Métrica} & \textbf{Baseline} & \textbf{NEOSYNT} & \textbf{Mejora} & \textbf{p-valor} \\
\midrule
Tasa de colapsos (\%) & $23.4 \pm 4.2$ & $2.97 \pm 0.8$ & -87.3\% & $<0.001$ \\
Latencia mediana (ms) & $145 \pm 28$ & $79 \pm 12$ & -45.5\% & $<0.001$ \\
Índice de estabilidad & $0.52 \pm 0.11$ & $0.89 \pm 0.04$ & +71.2\% & $<0.001$ \\
Robustez (CV error) & $0.38 \pm 0.08$ & $0.12 \pm 0.03$ & -68.4\% & $<0.001$ \\
\bottomrule
\end{tabular}
\end{table}

\subsection{Rendimiento bajo Carga Variable}

\begin{table}[H]
\centering
\caption{Rendimiento de NEOSYNT bajo diferentes condiciones de carga}
\begin{tabular}{lcccc}
\toprule
\textbf{Carga} & \textbf{Colapsos (\%)} & \textbf{Latencia (ms)} & \textbf{Estabilidad} & \textbf{CPU (\%)} \\
\midrule
Normal (1x) & $2.1 \pm 0.5$ & $72 \pm 8$ & $0.91 \pm 0.03$ & $12 \pm 3$ \\
Alta (2x) & $3.8 \pm 0.9$ & $89 \pm 15$ & $0.87 \pm 0.04$ & $24 \pm 5$ \\
Pico (5x) & $5.2 \pm 1.2$ & $124 \pm 22$ & $0.82 \pm 0.05$ & $45 \pm 8$ \\
\bottomrule
\end{tabular}
\end{table}

\subsection{Robustez ante Ruido}

\begin{table}[H]
\centering
\caption{Coeficiente de variación del error bajo diferentes niveles de ruido}
\begin{tabular}{lccc}
\toprule
\textbf{Ruido (varianza)} & \textbf{Baseline CV} & \textbf{NEOSYNT CV} & \textbf{Reducción} \\
\midrule
0.01 & 0.15 & 0.08 & -46.7\% \\
0.03 & 0.25 & 0.10 & -60.0\% \\
0.05 & 0.35 & 0.11 & -68.6\% \\
0.07 & 0.42 & 0.13 & -69.0\% \\
0.10 & 0.55 & 0.15 & -72.7\% \\
\bottomrule
\end{tabular}
\end{table}

\subsection{Efectividad del Gate de Consentimiento Bilateral}

\begin{table}[H]
\centering
\caption{Métricas de consentimiento bilateral}
\begin{tabular}{lcc}
\toprule
\textbf{Métrica} & \textbf{Sin Gate} & \textbf{Con Gate (140)} \\
\midrule
Interacciones exitosas (\%) & 67.3 & 94.2 \\
Colisiones/bloqueos (\%) & 15.8 & 1.2 \\
Tiempo medio de negociación (ms) & 89 & 34 \\
Consistencia de decisiones & 0.58 & 0.92 \\
\bottomrule
\end{tabular}
\end{table}

\subsection{Métricas de Integración (φ)}

\begin{table}[H]
\centering
\caption{Integración de información bajo diferentes condiciones}
\begin{tabular}{lcc}
\toprule
\textbf{Fase} & \textbf{φ medio} & \textbf{Identidad media} \\
\midrule
Exploración & $0.35 \pm 0.12$ & $0.42 \pm 0.15$ \\
Transición & $0.52 \pm 0.18$ & $0.58 \pm 0.12$ \\
Consolidación & $0.78 \pm 0.08$ & $0.85 \pm 0.06$ \\
Flow & $0.65 \pm 0.10$ & $0.72 \pm 0.09$ \\
Crisis & $0.18 \pm 0.15$ & $0.25 \pm 0.18$ \\
\bottomrule
\end{tabular}
\end{table}

\section{Gráficas de Resultados}

\begin{figure}[H]
\centering
\includegraphics[width=\textwidth]{../figuras/fig6_curvas_comparativas.pdf}
\caption{Comparativa de rendimiento: NEOSYNT vs Baseline}
\end{figure}

\section{Conclusiones Experimentales}

Los datos experimentales demuestran que el dispositivo NEOSYNT proporciona:

\begin{enumerate}
    \item \textbf{Reducción significativa de colapsos}: 87.3\% menos colapsos que la configuración baseline ($p < 0.001$).

    \item \textbf{Mejora de latencia}: 45.5\% de reducción en latencia mediana de coordinación.

    \item \textbf{Aumento de estabilidad}: Índice de estabilidad de 0.89 vs 0.52 del baseline (+71.2\%).

    \item \textbf{Robustez ante ruido}: El CV del error se mantiene por debajo de 0.15 incluso con varianza de ruido de 0.10.

    \item \textbf{Escalabilidad controlada}: Degradación gradual bajo carga (5.2\% colapsos en carga 5x vs 23.4\% del baseline en carga normal).
\end{enumerate}

Estos resultados validan la eficacia técnica de la combinación estructural de los elementos (100)-(170) del dispositivo NEOSYNT.

\end{document}
