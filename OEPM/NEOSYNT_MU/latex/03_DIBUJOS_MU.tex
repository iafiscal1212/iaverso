% 03_DIBUJOS_MU.tex - Dibujos para Modelo de Utilidad OEPM
\documentclass[11pt,a4paper]{article}

\usepackage[a-1b]{pdfx}
\usepackage[utf8]{inputenc}
\usepackage[T1]{fontenc}
\usepackage[spanish]{babel}
\usepackage{lmodern}
\usepackage[left=2cm,right=2cm,top=2cm,bottom=2cm]{geometry}
\usepackage{graphicx}
\usepackage{float}

\begin{document}

\begin{center}
{\LARGE\bfseries DIBUJOS}\\[1cm]
{\large Dispositivo de control autónomo multi-agente\\
con bus local y núcleo endógeno de seguridad (NEOSYNT)}\\[2cm]
\end{center}

% Figura 1
\begin{figure}[H]
\centering
\includegraphics[width=\textwidth]{../figuras/fig1_arquitectura.pdf}
\caption{Arquitectura general del dispositivo NEOSYNT mostrando los módulos (100)-(170)}
\end{figure}

\clearpage

% Figura 2
\begin{figure}[H]
\centering
\includegraphics[width=\textwidth]{../figuras/fig2_bus_buffers.pdf}
\caption{Bus local (100) y buffers circulares (110): secuencia de envío y validación}
\end{figure}

\clearpage

% Figura 3
\begin{figure}[H]
\centering
\includegraphics[width=\textwidth]{../figuras/fig3_nucleo_autonomo.pdf}
\caption{Núcleo autónomo (130): actualización del vector I en simplex mediante mirror-descent}
\end{figure}

\clearpage

% Figura 4
\begin{figure}[H]
\centering
\includegraphics[width=\textwidth]{../figuras/fig4_gate_consentimiento.pdf}
\caption{Gate de consentimiento bilateral (140): entradas y lógica de decisión}
\end{figure}

\clearpage

% Figura 5
\begin{figure}[H]
\centering
\includegraphics[width=\textwidth]{../figuras/fig5_watchdog_sandbox.pdf}
\caption{Watchdog (150) y sandbox (160): diagrama de estados y condiciones}
\end{figure}

\clearpage

% Figura 6
\begin{figure}[H]
\centering
\includegraphics[width=\textwidth]{../figuras/fig6_curvas_comparativas.pdf}
\caption{Curvas comparativas: colapsos, latencia, estabilidad y robustez}
\end{figure}

\clearpage

% Lista de referencias numéricas
\section*{Lista de Referencias Numéricas}

\begin{tabular}{ll}
\textbf{(100)} & Bus local (UNIX socket) \\
\textbf{(110)} & Buffers circulares \\
\textbf{(120)} & Módulo de validación (checksum + log inmutable) \\
\textbf{(130)} & Núcleo autónomo (vector $\mathbf{I}$, mirror-descent, ruido OU) \\
\textbf{(140)} & Gate de consentimiento bilateral \\
\textbf{(150)} & Watchdog de recursos \\
\textbf{(160)} & Sandbox de evolución de código \\
\textbf{(170)} & Planificador (colas/cache) \\
\end{tabular}

\end{document}
