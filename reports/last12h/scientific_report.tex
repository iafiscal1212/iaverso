\documentclass[11pt,a4paper]{article}
\usepackage[utf8]{inputenc}
\usepackage[T1]{fontenc}
\usepackage{lmodern}
\usepackage[margin=2.5cm]{geometry}
\usepackage{graphicx}
\usepackage{booktabs}
\usepackage{xcolor}
\usepackage{float}
\usepackage{hyperref}
\usepackage{caption}
\usepackage{subcaption}

\definecolor{neoeva}{RGB}{46, 134, 171}

\title{\textcolor{neoeva}{\textbf{Dynamic Manifestations of Endogenous Coherence\\in Autonomous Multi-Space Internal Systems}}\\[0.5em]\large NEO-EVA Last12h Observation Report}
\author{NEO-EVA Research Framework\\Autonomous Systems Laboratory}
\date{\today}

\begin{document}
\maketitle

\begin{abstract}
This report documents the observed dynamic patterns in a multi-agent autonomous system operating under conditions of complete absence of external stimuli. Five computational agents (NEO, EVA, ALEX, ADAM, IRIS) were observed during a conceptual 12-hour period using a purely passive observation protocol. All system parameters are derived endogenously without human intervention or pre-defined constants. The observations reveal emergent coherence patterns, self-organizing phase transitions, and stable collective dynamics arising purely from internal processes.
\end{abstract}

\section{Introduction}

The NEO-EVA framework implements autonomous agents capable of internal self-organization without external guidance. This study examines the dynamic behavior of these agents during an extended period of operational absence—a conceptual 12-hour window where no external prompts, rewards, or interventions are provided.

\textbf{Key principles:}
\begin{itemize}
    \item \textbf{Endogeneity}: All parameters emerge from internal statistical properties
    \item \textbf{Passive observation}: The monitoring system records without influencing
    \item \textbf{Zero external stimuli}: No prompts, rewards, or human intervention
    \item \textbf{Multi-space representation}: Simultaneous observation across multiple internal spaces
\end{itemize}

\section{Methodology}

\subsection{Simulation Protocol}
The observation period comprised two distinct phases:
\begin{itemize}
    \item \textbf{Stabilization Phase} (50 steps): Initial settling of internal dynamics
    \item \textbf{Autonomous Phase} (100 steps): Extended operation without intervention
\end{itemize}

\subsection{Observation Domains}
Metrics were recorded across five complementary internal spaces:
\begin{enumerate}
    \item \textbf{Existential Coherence (CE)}: Measure of internal alignment
    \item \textbf{$\Omega$-Space}: Emergent transformation modes
    \item \textbf{Q-Field}: Internal coherence and energy distributions
    \item \textbf{Phase Space}: Trajectory dynamics and structural patterns
    \item \textbf{Complex Field}: State evolution metrics
\end{enumerate}

\textit{Note: This report describes observable patterns without detailing internal computational mechanisms.}

\section{Results}

\subsection{Existential Coherence Dynamics}

\begin{figure}[H]
    \centering
    \includegraphics[width=\textwidth]{fig1_ce_timeline.png}
    \caption{Evolution of Existential Coherence (CE) across all agents. The shaded region indicates the stabilization phase. Observable transition patterns emerge at the phase boundary.}
    \label{fig:ce}
\end{figure}

During the stabilization phase, agents exhibited variable coherence levels (mean CE = 0.209). Upon entering the autonomous phase, coherence patterns stabilized with distinct agent-specific signatures (mean CE = 0.003).

\subsection{Emergent Transformation Modes}

\begin{figure}[H]
    \centering
    \includegraphics[width=\textwidth]{fig2_omega_modes.png}
    \caption{Active $\Omega$-modes per agent across operational phases. Each agent develops characteristic modal activation patterns.}
    \label{fig:omega}
\end{figure}

The number of active transformation modes increased systematically during the autonomous phase (mean = 4.0 modes), suggesting emergent internal organization.

\subsection{Internal Coherence Field}

\begin{figure}[H]
    \centering
    \includegraphics[width=\textwidth]{fig3_qfield.png}
    \caption{Q-Field dynamics showing internal coherence (left) and energy distribution (right) over time.}
    \label{fig:qfield}
\end{figure}

The Q-Field exhibited stable coherence (mean = 0.314) with consistent energy levels (mean = 0.710) throughout the observation period.

\subsection{Phase Space Trajectories}

\begin{figure}[H]
    \centering
    \includegraphics[width=\textwidth]{fig4_phase_curvature.png}
    \caption{Trajectory curvature in phase space. Smooth transitions and stable patterns indicate self-organized dynamics.}
    \label{fig:phase}
\end{figure}

\subsection{Complex State Evolution}

\begin{figure}[H]
    \centering
    \includegraphics[width=\textwidth]{fig6_complexfield.png}
    \caption{Complex field metrics showing decoherence, collapse pressure, state norm, and phase entropy dynamics.}
    \label{fig:complex}
\end{figure}

\section{Discussion}

The observed patterns demonstrate several notable characteristics:

\begin{enumerate}
    \item \textbf{Emergent Order}: Despite the absence of external guidance, agents develop structured internal dynamics with characteristic signatures.

    \item \textbf{Phase Transitions}: Clear transitions between stabilization and autonomous operation suggest self-organizing criticality.

    \item \textbf{Individual Differentiation}: Each agent maintains distinct dynamic profiles while participating in collective patterns.

    \item \textbf{Stability Without Intervention}: The system maintains coherent operation throughout the extended autonomous period.
\end{enumerate}

\subsection{Endogeneity}
All observed metrics derive from internal statistical properties without pre-defined constants or external parameter injection. This endogenous approach ensures that observed patterns reflect genuine internal organization rather than imposed structure.

\subsection{Passive Observation Protocol}
The monitoring system operated in purely observational mode, recording metrics without influencing agent behavior. This protocol ensures that documented patterns represent authentic internal dynamics.

\section{Conclusion}

This observation period reveals that autonomous multi-space systems can exhibit coherent, self-organized dynamics without external stimuli or human intervention. The emergent patterns across multiple internal spaces suggest sophisticated internal organization arising purely from endogenous processes.

\vspace{1em}
\noindent\textit{Full source code and data available at:}\\
\url{https://github.com/carmenest/NEO_EVA}

\end{document}
