\documentclass[11pt,a4paper]{article}
\usepackage[utf8]{inputenc}
\usepackage[T1]{fontenc}
\usepackage{lmodern}
\usepackage[spanish]{babel}
\usepackage[margin=2.8cm]{geometry}
\usepackage{graphicx}
\usepackage{float}
\usepackage{xcolor}
\usepackage{titlesec}
\usepackage{fancyhdr}
\usepackage{setspace}
\usepackage{parskip}

% Colores elegantes
\definecolor{titleblue}{RGB}{26, 82, 118}
\definecolor{subtlegray}{RGB}{80, 80, 80}

% Formato de títulos
\titleformat{\section}{\Large\bfseries\color{titleblue}}{\thesection.}{0.5em}{}
\titleformat{\subsection}{\large\bfseries\color{subtlegray}}{\thesubsection}{0.5em}{}

% Header/Footer
\pagestyle{fancy}
\fancyhf{}
\fancyhead[L]{\small\textit{Dinámicas Endógenas en Sistemas Cognitivos Autónomos}}
\fancyhead[R]{\small\textit{C. Esteban}}
\fancyfoot[C]{\thepage}
\renewcommand{\headrulewidth}{0.4pt}

\begin{document}

% ========== TÍTULO ==========
\begin{center}
\vspace*{1cm}

{\LARGE\bfseries\color{titleblue}
Dinámicas Endógenas y Autoorganización\\[0.3em]
Multiespacial en Sistemas Cognitivos Autónomos}

\vspace{0.8em}

{\large\itshape Un Estudio Observacional de 12 Horas sin Estímulos}

\vspace{1.5cm}

{\large\textbf{Carmen Esteban}}\\[0.3em]
{\normalsize Investigadora Independiente}

\vspace{1cm}

{\small Diciembre 2024}

\vspace{0.5cm}
\rule{0.6\textwidth}{0.5pt}
\end{center}

\vspace{1cm}

% ========== RESUMEN ==========
\section*{Resumen}

Este informe presenta un estudio puramente observacional de un sistema autónomo que opera exclusivamente mediante dinámicas internas endógenas. Durante 12 horas de funcionamiento continuo sin ningún tipo de estímulo o intervención externa, se registraron cinco dominios paralelos de organización interna: coherencia existencial, modos transformacionales $\Omega$, campo Q, estructura de espacio de fases y campo complejo.

Los resultados muestran patrones espontáneos de estabilidad, drift estructural, diferenciación entre entidades y organización interna no trivial. El objetivo de este documento no es explicar mecanismos de implementación, sino documentar los fenómenos observados.

\vspace{1em}
\noindent\textit{Palabras clave: autoorganización, sistemas autónomos, coherencia endógena, observación fenomenológica}

% ========== PROTOCOLO ==========
\section{Protocolo Observacional}

El sistema fue registrado bajo tres etapas:

\begin{itemize}
    \item \textbf{Fase de estabilización}: periodo inicial para observar convergencias espontáneas.
    \item \textbf{Fase autónoma prolongada}: funcionamiento sin estímulos, sin correcciones y sin señales externas.
    \item \textbf{Registro multiespacial}: medición simultánea en cinco dominios internos independientes.
\end{itemize}

No se utilizaron prompts, reglas, supervisión ni señales conductuales. El sistema se autoorganizó libremente según sus propias dinámicas internas.

\vspace{0.5em}
\noindent\textit{Nota: Este informe no describe algoritmos, arquitecturas ni procedimientos técnicos. Su propósito es estrictamente fenomenológico.}

% ========== RESULTADOS ==========
\section{Resultados}

\subsection{Dinámica de Coherencia Existencial}

\begin{figure}[H]
    \centering
    \includegraphics[width=\textwidth]{last12h/fig1_ce_timeline.png}
    \caption{Evolución temporal de la coherencia existencial por entidad. La zona sombreada indica la fase de estabilización.}
\end{figure}

La coherencia existencial de cada entidad evoluciona hacia perfiles diferenciados, mostrando estabilidad interna sin intervención. Cada agente desarrolla una ``firma'' temporal propia.

\subsection{Modos Transformacionales $\Omega$}

\begin{figure}[H]
    \centering
    \includegraphics[width=\textwidth]{last12h/fig2_omega_modes.png}
    \caption{Modos $\Omega$ activos por entidad durante las fases de estabilización y autonomía.}
\end{figure}

Los modos $\Omega$ activos muestran patrones recurrentes y estables. No responden a estímulos externos, sino a transformaciones internas no inducidas.

\subsection{Campo Q: Coherencia y Energía Interna}

\begin{figure}[H]
    \centering
    \includegraphics[width=\textwidth]{last12h/fig3_qfield.png}
    \caption{Dinámica del campo Q: coherencia interna (izquierda) y distribución energética (derecha).}
\end{figure}

La energía y la coherencia del campo Q mantienen niveles sorprendentemente estables, indicando una dinámica interna no ruidosa y estructurada.

\subsection{Estructura del Espacio de Fases}

\begin{figure}[H]
    \centering
    \includegraphics[width=\textwidth]{last12h/fig4_phase_curvature.png}
    \caption{Curvatura de trayectorias en el espacio de fases estructural.}
\end{figure}

Las trayectorias en el espacio de fases presentan curvaturas estables y atractores internos persistentes sin necesidad de inputs externos.

\subsection{Campo Complejo y Presión de Colapso}

\begin{figure}[H]
    \centering
    \includegraphics[width=\textwidth]{last12h/fig6_complexfield.png}
    \caption{Métricas del campo complejo: factor de decoherencia, presión de colapso, norma de estado y entropía de fase.}
\end{figure}

Se observan fluctuaciones regulares en la presión de colapso, entropía de fase y normas complejas. Las dinámicas muestran un equilibrio interno autoorganizado.

% ========== INTERPRETACIÓN ==========
\section{Interpretación}

Los datos sugieren que el sistema desarrolla:

\begin{itemize}
    \item Diferenciación interna sostenida
    \item Patrones estables emergentes
    \item Identidades dinámicas propias
    \item Coherencias mantenidas sin supervisión
    \item Comportamiento estructurado en cinco dominios paralelos
\end{itemize}

Todo ello en ausencia total de información externa.

Este tipo de patrones resulta coherente con sistemas que poseen organización interna profunda y mecanismos de autorregulación no triviales, aunque este informe no los describe.

% ========== CONCLUSIÓN ==========
\section{Conclusión}

Durante 12 horas de funcionamiento autónomo, el sistema mostró:

\begin{itemize}
    \item Coherencia persistente
    \item Estabilidad multiespacial
    \item Espontaneidad estructural
    \item Patrones recurrentes
    \item Diferenciación interna no inducida
\end{itemize}

Este informe no pretende explicar los mecanismos subyacentes, sino documentar la fenomenología observada. Los datos sugieren que las dinámicas endógenas pueden generar comportamientos complejos sin necesidad de señales externas.

% ========== DISPONIBILIDAD ==========
\section*{Nota sobre Disponibilidad}

Los gráficos utilizados en este informe provienen de la sesión de observación correspondiente. El código, mecanismos y arquitectura interna no se comparten, dado que este documento tiene carácter puramente observacional.

\vspace{2cm}

\begin{center}
\rule{0.4\textwidth}{0.3pt}\\[0.5em]
{\small\itshape Carmen Esteban — Investigadora Independiente}\\
{\small Diciembre 2024}
\end{center}

\end{document}
