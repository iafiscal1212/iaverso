\documentclass[11pt,a4paper]{article}
\usepackage[utf8]{inputenc}
\usepackage[spanish]{babel}
\usepackage{amsmath,amssymb,amsfonts}
\usepackage{graphicx}
\usepackage{booktabs}
\usepackage{hyperref}
\usepackage{xcolor}
\usepackage{float}
\usepackage{algorithm}
\usepackage{algpseudocode}
\usepackage{geometry}
\geometry{margin=2.5cm}

\title{\textbf{NEO$\leftrightarrow$EVA: Sistema Dual de Agentes con Dinámica 100\% Endógena y Acoplamiento por Consentimiento}}

\author{Carmen Esteban\\
\small Investigación en Sistemas Dinámicos Endógenos\\
\small \texttt{github.com/carmenest/NEO\_EVA}}

\date{29 de Noviembre de 2025}

\begin{document}

\maketitle

\begin{abstract}
Presentamos NEO$\leftrightarrow$EVA, un sistema dual de agentes autónomos que mantienen vectores de intención en el simplex y pueden acoplarse mediante un bus de comunicación. La contribución principal es metodológica: \textbf{todos los parámetros operativos se derivan exclusivamente de la historia propia del sistema} (cuantiles, IQR, $\sigma$, escalado $1/\sqrt{T}$), eliminando constantes arbitrarias. Además, implementamos un mecanismo de \textbf{consentimiento bilateral} donde cada agente decide autónomamente si desea acoplarse, utilizando un bandit de 3 brazos para aprender el modo óptimo de interacción. El sistema pasa todas las auditorías de endogeneidad con 0 violaciones y demuestra comportamiento emergente de cooperación sin órdenes externas.

\textbf{Palabras clave:} sistemas dinámicos, endogeneidad, simplex, mirror descent, acoplamiento causal, consentimiento, Thompson sampling
\end{abstract}

\section{Introducción}

\subsection{Motivación: El Problema de los ``Números Mágicos''}

En la literatura de sistemas dinámicos y aprendizaje por refuerzo es común encontrar constantes arbitrarias que afectan críticamente el comportamiento: tasas de aprendizaje fijas ($\eta = 0.01$), umbrales hardcodeados ($threshold = 0.5$), límites de clip ($\pm 5$), y factores multiplicativos sin justificación ($boost \times 2.0$).

Estos ``números mágicos'' presentan problemas fundamentales:
\begin{itemize}
    \item \textbf{Falta de justificación teórica}: ¿Por qué 0.01 y no 0.001?
    \item \textbf{Dependencia del dominio}: Funcionan para un problema pero fallan en otro
    \item \textbf{Reproducibilidad cuestionable}: Pequeños cambios producen resultados muy diferentes
    \item \textbf{P-hacking implícito}: Se pueden ajustar hasta obtener resultados deseados
\end{itemize}

\subsection{Principio Rector: Endogeneidad Total}

Proponemos un principio metodológico estricto:

\begin{quote}
\textit{``Si no sale de la historia, no entra en la dinámica''}
\end{quote}

Esto significa que todo parámetro numérico que afecte la dinámica debe derivarse de:
\begin{itemize}
    \item Cuantiles de la historia observada ($p_{50}, p_{75}, p_{95}, p_{99}$)
    \item Estadísticas de dispersión (IQR, $\sigma$, MAD)
    \item Escalados teóricos ($1/\sqrt{T}$, $\log T$)
    \item Constantes geométricas ($1/\sqrt{2}, 1/\sqrt{3}, 1/\sqrt{12}$)
\end{itemize}

Las únicas constantes permitidas son tolerancias numéricas ($\varepsilon = 10^{-12}$) para evitar divisiones por cero.

\subsection{¿Qué son NEO y EVA?}

NEO y EVA son dos agentes que representan diferentes ``mundos'' o perspectivas. Cada uno mantiene un vector de intención $\mathbf{I} \in \Delta^2$ (simplex 2D), que representa una distribución sobre tres modos operativos:
\begin{itemize}
    \item \textbf{S (Sintaxis)}: Enfoque en estructura y forma
    \item \textbf{N (Novedad)}: Exploración y creatividad
    \item \textbf{C (Coherencia)}: Consistencia y estabilidad
\end{itemize}

Los agentes evolucionan independientemente pero pueden acoplarse a través de un BUS de comunicación, donde cada uno observa resúmenes estadísticos del otro y decide si incorporarlos a su dinámica.

\subsection{Objetivo del Proyecto}

El objetivo es crear un sistema donde:
\begin{enumerate}
    \item Toda la dinámica sea \textbf{100\% endógena}
    \item El acoplamiento sea \textbf{por consentimiento bilateral}
    \item Los agentes \textbf{aprendan} cómo acoplarse óptimamente
    \item El comportamiento emergente sea \textbf{verificable y reproducible}
\end{enumerate}

\section{Métodos}

\subsection{Estado de Intención en el Simplex}

El estado de intención $\mathbf{I}_t = (S_t, N_t, C_t)$ vive en el simplex:
\begin{equation}
\Delta^2 = \{\mathbf{I} : \sum_i I_i = 1, I_i \geq 0\}
\end{equation}

La actualización se realiza mediante \textbf{mirror descent} con entropía negativa como función de Bregman:
\begin{equation}
\mathbf{I}_{t+1} = \text{softmax}(\log \mathbf{I}_t + \eta_t \boldsymbol{\Delta}_t)
\end{equation}

donde $\eta_t$ es la tasa de aprendizaje (endógena) y $\boldsymbol{\Delta}_t$ es la dirección de actualización en el plano tangente.

El plano tangente del simplex se parametriza con dos vectores ortonormales:
\begin{align}
\mathbf{u}_1 &= (1, -1, 0) / \sqrt{2} \\
\mathbf{u}_2 &= (1, 1, -2) / \sqrt{6}
\end{align}

\subsection{Parámetros Endógenos}

\subsubsection{Tamaño de Ventana}
\begin{equation}
w = \max\{10, \lfloor\sqrt{T}\rfloor\}
\end{equation}

\subsubsection{Tasa de Aprendizaje $\tau$}
\begin{equation}
\tau = \frac{\text{IQR}(r)}{\sqrt{T}} \times \frac{\sigma_{med}}{\text{IQR}_{hist} + \varepsilon}
\end{equation}

con piso $\tau_{floor} = \sigma_{med} / T$.

\subsubsection{Gate Crítico}
\begin{equation}
\text{Gate activo} \Leftrightarrow \rho \geq \rho_{p95} \wedge \text{IQR} \geq \text{IQR}_{p75}
\end{equation}

donde los umbrales son \textbf{cuantiles puros} de la historia, sin factores multiplicativos.

\subsection{Variabilidad OU Endógena}

La exploración se modela con un proceso de Ornstein-Uhlenbeck bidimensional $\mathbf{Z}_t$ en el plano tangente:
\begin{equation}
d\mathbf{Z}_t = -\theta \mathbf{Z}_t \, dt + \sigma \, d\mathbf{W}_t
\end{equation}

\textbf{Todos los parámetros son endógenos:}

\begin{table}[H]
\centering
\begin{tabular}{ll}
\toprule
Parámetro & Fórmula \\
\midrule
$\theta_{floor}$ & $\sigma_{med} / T$ \\
$\theta_{ceil}$ & $\text{quantile}(\theta_{history}, p_{99})$ \\
$\sigma$ & $\sqrt{\tau}$ \\
Límites & $\text{clip}(\mathbf{Z}, q_{0.001}, q_{0.999})$ o $m \pm 4 \times \text{MAD}$ \\
\bottomrule
\end{tabular}
\end{table}

\subsection{Acoplamiento $\kappa$ Endógeno}

El acoplamiento entre mundos se calcula como:
\begin{equation}
\kappa = \frac{u_Y}{1+u_X} \times \frac{\lambda_1^Y}{\lambda_1^Y + \lambda_1^X + \varepsilon} \times \frac{\text{conf}^Y}{1 + \text{CV}(r^X)}
\end{equation}

donde:
\begin{itemize}
    \item $u$ = incertidumbre = IQR$(r)/\sqrt{T}$
    \item $\lambda_1$ = primer autovalor de cov$(\mathbf{I})$ (dominancia direccional)
    \item conf = confianza = $\max(\mathbf{I}) - \text{segundo\_max}(\mathbf{I})$
    \item CV = coeficiente de variación de residuos
\end{itemize}

\textbf{Importante}: $\kappa$ es 100\% estadístico, sin factores arbitrarios.

\subsection{Phase 7: Consentimiento Bilateral}

\subsubsection{Beneficio Esperado}
\begin{equation}
\Delta\hat{U}_t^{X \leftarrow Y} = \frac{u_t^Y}{1+u_t^X} \times \frac{\lambda_1^Y}{\lambda_1^Y + \lambda_1^X + \varepsilon} \times \frac{\text{conf}_t^Y}{1 + \text{CV}(r_t^X)}
\end{equation}

Normalizado por cuantiles históricos $\Rightarrow \in [0, 1]$.

\subsubsection{Coste Endógeno}
\begin{equation}
\text{coste}_t^X = \text{Rank}\left(\mathbf{1}\{\rho(J_t^X) \geq p_{95}\} + \text{RankInvVar}(I) + \text{Rank}(\text{latencia})\right) / 3
\end{equation}

\subsubsection{Voluntad Individual}
\begin{equation}
\pi_t^X = \sigma\left(\text{rank}(\Delta\hat{U}) - \text{rank}(\text{coste})\right)
\end{equation}

Decisión estocástica: $a_t^X \sim \text{Bernoulli}(\pi_t^X)$

\subsubsection{Consentimiento Bilateral}
\begin{equation}
\text{Acoplamiento activo} \Leftrightarrow a_t^{NEO} = 1 \wedge a_t^{EVA} = 1
\end{equation}

\textbf{Sin consentimiento de ambos, no hay conexión.}

\subsubsection{Modo Autodeterminado}

Cada agente ejecuta un bandit de 3 brazos (Thompson Sampling):
\begin{itemize}
    \item $m = -1$: Anti-alineado (exploración contraria)
    \item $m = 0$: Off (sin acoplamiento)
    \item $m = +1$: Alineado (cooperación)
\end{itemize}

Recompensa: $G_t = \text{BordaRank}(\Delta\text{RMSE}, \Delta\text{MDL}, \text{MI})$

\subsubsection{Stopping Rules Endógenas}

\begin{itemize}
    \item Cortar si $\rho(J_t^X) \geq p_{99}(\rho^X)$ (tensión crítica)
    \item Cortar si $\text{Var}_w(\mathbf{I}^X) \leq p_{25}$ (pérdida de exploración)
    \item Cortar si regret del bandit empeora por debajo de $p_{50}$
\end{itemize}

\section{Auditoría de Endogeneidad}

Implementamos un sistema de auditoría con tres módulos:

\begin{enumerate}
    \item \textbf{Auditoría Estática}: Escaneo de código buscando literales numéricos sospechosos
    \item \textbf{Auditoría Dinámica}: Tests de invariancia (escalado $1/\sqrt{T}$, sensibilidad a varianza)
    \item \textbf{Auditoría de Acoplamiento}: Verificación de que $\kappa$ no usa constantes mágicas
\end{enumerate}

\begin{table}[H]
\centering
\begin{tabular}{lll}
\toprule
Módulo & Estado & Detalles \\
\midrule
Auditoría Estática & PASS & 0 violaciones de 252 hallazgos \\
Auditoría Dinámica & PASS & 2/2 tests de invariancia \\
Auditoría $\kappa$ & PASS & 5 ejemplos, sin magia \\
\textbf{Estado Global} & \textbf{GO} & Listo para publicación \\
\bottomrule
\end{tabular}
\caption{Resultados de auditoría de endogeneidad}
\end{table}

\section{Resultados}

\subsection{Corrida Larga (20,000 ciclos)}

\begin{figure}[H]
\centering
\includegraphics[width=0.8\textwidth]{figures/fig1_longrun_correlation.png}
\caption{Evolución de la correlación NEO$\leftrightarrow$EVA durante 20,000 ciclos. La correlación comienza alta ($\sim$0.57) por efectos transitorios, decae gradualmente, y converge a $\sim$0 reflejando exploración independiente.}
\end{figure}

La correlación evoluciona de forma característica:
\begin{itemize}
    \item \textbf{Warmup (0-2k)}: Correlación alta ($\sim$0.57) por efectos transitorios
    \item \textbf{Estabilización (2k-10k)}: Decaimiento gradual hacia $\sim$0.3
    \item \textbf{Equilibrio (10k-20k)}: Convergencia a $\sim$0, reflejando exploración independiente
\end{itemize}

\subsection{Activaciones de Acoplamiento}

\begin{figure}[H]
\centering
\includegraphics[width=0.8\textwidth]{figures/fig2_coupling_activations.png}
\caption{Activaciones de acoplamiento. NEO: 758 activaciones (3.8\%), EVA: 6,374 activaciones (31.9\%). La asimetría refleja las diferentes condiciones de gate en cada mundo.}
\end{figure}

\subsection{Comparación v1 (hardcoded) vs v2 (endógeno)}

\begin{table}[H]
\centering
\begin{tabular}{lccc}
\toprule
Métrica & v1 (hardcoded) & v2 (endógeno) & v2 ablación \\
\midrule
Correlación media & 0.35 & -0.07 & 0.01 \\
Activaciones NEO (\%) & 27.6 & 3.8 & 0 \\
Activaciones EVA (\%) & 29.0 & 31.9 & 0 \\
Varianza NEO & 0.267 & 0.282 & 0.356 \\
Varianza EVA & 0.210 & 0.609 & 0.289 \\
\bottomrule
\end{tabular}
\caption{Comparación entre versiones. La ablación ($\kappa=0$) muestra correlación $\sim$0, confirmando que el acoplamiento es causal.}
\end{table}

\subsection{Phase 7: Resultados de Consentimiento (5,000 ciclos)}

\begin{table}[H]
\centering
\begin{tabular}{lcc}
\toprule
Métrica & NEO & EVA \\
\midrule
Propuestas de consentimiento & $\sim$25\% & $\sim$25\% \\
Consentimientos bilaterales & 1,231 & 1,227 \\
Ratio bilateral & 24.6\% & 24.5\% \\
\bottomrule
\end{tabular}
\caption{Métricas de consentimiento bilateral}
\end{table}

\begin{table}[H]
\centering
\begin{tabular}{lccc}
\toprule
Modo & NEO & EVA & Descripción \\
\midrule
-1 (anti-align) & 619 (12.4\%) & 586 (11.7\%) & Exploración contraria \\
0 (off) & 3,803 (76.1\%) & 3,821 (76.4\%) & Sin acoplamiento \\
+1 (align) & 578 (11.6\%) & 593 (11.9\%) & Alineación \\
\bottomrule
\end{tabular}
\caption{Distribución de modos aprendida por el bandit}
\end{table}

\textbf{Interpretación}: El sistema aprende a usar los tres modos. El modo ``off'' (0) domina naturalmente cuando el gate está cerrado o no hay consentimiento bilateral. La distribución no degenerada ($\sim$12\% para cada modo activo) indica aprendizaje efectivo.

\subsection{Comparación Coupled vs Ablation}

\begin{table}[H]
\centering
\begin{tabular}{lcc}
\toprule
Métrica & Coupled & Ablation \\
\midrule
Correlación media & -0.030 & 0.009 \\
Eventos bilaterales & 1,231 & 310 \\
\bottomrule
\end{tabular}
\caption{El sistema coupled tiene $\sim$4x más eventos bilaterales, demostrando que el acoplamiento tiene efecto causal.}
\end{table}

\subsection{Trayectorias en el Simplex}

\begin{figure}[H]
\centering
\includegraphics[width=0.7\textwidth]{figures/fig6_simplex_trajectory.png}
\caption{Trayectorias de NEO y EVA en el simplex. Las trayectorias muestran exploración del espacio sin quedarse atrapadas en las esquinas.}
\end{figure}

\section{Discusión}

\subsection{Por Qué la Endogeneidad Mejora la Validez}

\begin{enumerate}
    \item \textbf{Reproducibilidad}: Los resultados dependen solo de la historia, no de decisiones del investigador
    \item \textbf{Generalización}: El sistema se adapta automáticamente a diferentes escalas y dominios
    \item \textbf{Transparencia}: Cada número tiene una derivación explícita y verificable
    \item \textbf{Resistencia a p-hacking}: No hay parámetros que ajustar para obtener resultados deseados
\end{enumerate}

\subsection{Emergencia del Consentimiento}

El mecanismo de consentimiento bilateral produce comportamiento cooperativo \textbf{sin órdenes externas}:
\begin{itemize}
    \item Cada agente calcula independientemente su voluntad de acoplarse
    \item Solo hay interacción cuando ambos la desean
    \item El modo de interacción se aprende mediante experiencia propia
    \item Las reglas de parada protegen contra inestabilidad
\end{itemize}

Esto representa un modelo de ``cooperación ética'' donde la autonomía de cada agente se respeta.

\subsection{Limitaciones}

\begin{enumerate}
    \item \textbf{Dependencia de ventana}: $w = \max\{10, \lfloor\sqrt{T}\rfloor\}$ es una elección funcional específica
    \item \textbf{Estacionariedad}: El sistema asume que las estadísticas pasadas son informativas del futuro
    \item \textbf{Warmup}: Durante $T < w$, se usa $\sigma_{uniform} = 1/\sqrt{12}$ como prior de máxima entropía
\end{enumerate}

\subsection{Trabajo Futuro}

\begin{enumerate}
    \item \textbf{Interfaz con tarea externa}: Conectar $\mathbf{I}_t$ con recompensas o métricas de desempeño
    \item \textbf{Multi-agente}: Extender a más de dos mundos
    \item \textbf{Adaptación no estacionaria}: Detectar cambios de régimen
    \item \textbf{Comunicación rica}: Más allá de resúmenes estadísticos
\end{enumerate}

\section{Conclusión}

Hemos demostrado que es posible construir un sistema dinámico dual completamente endógeno, donde:

\begin{enumerate}
    \item \textbf{Cada parámetro numérico} se deriva de la historia observada sin constantes arbitrarias
    \item \textbf{El acoplamiento} ocurre por consentimiento bilateral, respetando la autonomía de cada agente
    \item \textbf{El modo de interacción} se aprende mediante Thompson sampling
    \item \textbf{La auditoría} confirma 0 violaciones de endogeneidad
\end{enumerate}

El principio \textbf{``si no sale de la historia, no entra en la dinámica''} no solo es metodológicamente superior (transparente, reproducible, sin p-hacking), sino que también produce inferencias más confiables y comportamiento emergente de cooperación.

NEO y EVA demuestran que es posible crear agentes que negocian su interacción de forma autónoma y ética, sin órdenes externas ni constantes mágicas.

\section*{Reproducibilidad}

Todo el código está disponible en: \url{https://github.com/carmenest/NEO_EVA}

\begin{table}[H]
\centering
\begin{tabular}{ll}
\toprule
Archivo & SHA256 \\
\midrule
phase6\_coupled\_system\_v2.py & 47ab6020... \\
phase7\_consent\_autocouple.py & 6e4741fb... \\
endogeneity\_auditor.py & fa23739a... \\
\bottomrule
\end{tabular}
\end{table}

\section*{Referencias}

\begin{enumerate}
    \item Beck, A., \& Teboulle, M. (2003). Mirror descent and nonlinear projected subgradient methods for convex optimization. \textit{Operations Research Letters}, 31(3), 167-175.

    \item Kullback, S., \& Leibler, R. A. (1951). On information and sufficiency. \textit{Annals of Mathematical Statistics}, 22(1), 79-86.

    \item Schreiber, T. (2000). Measuring information transfer. \textit{Physical Review Letters}, 85(2), 461.

    \item Thompson, W. R. (1933). On the likelihood that one unknown probability exceeds another in view of the evidence of two samples. \textit{Biometrika}, 25(3-4), 285-294.
\end{enumerate}

\end{document}
